\documentclass[10pt]{article} \usepackage[a4paper,left=0.8in,right=0.8in,top=0.4in,bottom=0.8in]{geometry} \usepackage{graphicx} 
\usepackage{url} 
\usepackage{datetime} 

\title{\textbf{CS296: Group 1 Project}\\{Gear Stop Clock driven by Perpetual Motion Machine}}
\author{ Siddharth Patel\\ 
120050001\\ 
\texttt{sidd@cse.iitb.ac.in} 
\and Ramprakash K\\ 
120050083\\ 
\texttt{ramprakash@cse.iitb.ac.in} 
\and Viplov Jain\\ 
120050084\\ 
\texttt{viplov@cse.iitb.ac.in}\\ 
} 
\date{\today} 

\begin{document} 


\bibliographystyle{plain} 
\maketitle 


\section{Introduction}
This report explains our project for cs296 lab course. It contains a system of gears kept in motion by a perpetual wheel.
\section{Physics behind the simulation.}
\subsection{Perpetual Wheel}
\subsubsection{Perpetual Motion}
Perpetual motion is the action of the device that once set in motion would continue in motion forever, with no additional energy required to maintain it. Such devices are impossible in real life due to energy loses due to friction. 
\subsubsection{Perpetual Wheel}
Wheel we used in this project is the modified version of the Bhaskara Wheel which was one of the most earliest recorded man-made perpetual wheel. 
\section{Original Design}
*****
\section{Final Design and Changes made in the original design}
*****
\section{Reasons for modification of our original design}
*****
\section{Analysis of code}
*****
\section{Graphs}

\bibliographystyle{ }
\bibliography{ }

\end{document}
